\documentclass[11pt,twoside,openright]{report}
\usepackage{a4}
\setlength{\parindent}{0in} % fjerner indents

\usepackage[utf8]{inputenc}
\usepackage[english]{babel}

\usepackage{amsmath}
\usepackage{amsfonts}
\usepackage{amssymb}
\usepackage{array}

\usepackage{float}
\usepackage{appendix}

\usepackage{graphicx}

\usepackage{vmargin}
\setmargrb{2.75cm}{1.75cm}{2.25cm}{2.25cm} % final print
%\setmargrb{2.75cm}{1.75cm}{3.25cm}{2.25cm} % todo print


\usepackage{hyperref}
\hypersetup{
	pdfborder = {0 0 0}
}

\usepackage[%
	per-mode = symbol,
	exponent-product = \cdot,
	complex-root-position = before-number,
	output-complex-root = \text{j},
	group-digits = true,
	binary-units = true
]{siunitx}


\usepackage{enumitem}

% Til kode eksempler:
\usepackage{color}
\usepackage{listings}

\usepackage{setspace}

\def\lstlistingname{Code snippet}           % Definerer hvad der står foran et stykke kodes caption
\lstset{
	basicstyle=\footnotesize\ttfamily,    % Lille skrifttype
	keywordstyle=\color{blue}\bfseries,   % Keywords blå og bold
	commentstyle=\color[RGB]{34,139,34},  % Default comments mørkegrøn
	showstringspaces=false,               % Ingen symbol for mellemrum i strings
	numbers=left,                         % Linjenumre til venstre
	numberstyle=\tiny\color{darkgray},    % Små tal på linjenumre med farve skrift
	numbersep=5pt,                        % Afstand fra linjenummer og ind til kode
	backgroundcolor=\color{codesnippetbg},% Bg farve
	tabsize=2,                            % Indenteringer = 4 spaces
	columns=fixed,                   	  % Kan give problemer med bredde på bogstaver men skulle ikke da vi bruger ttfamily
	breaklines=true,                      % Deler en for lang linje over to linjer
	frame=tb,                		      % Styrer hvor streger skal placeres
	captionpos=t,                         % Caption til kode under og over kodeeksemplet
	rulecolor=\color{black},			  % Farven på frame
	escapeinside={(*@}{@*)},              % Giver mulighed for at lave en (*@\label{label}@*), på en kodelinje, så man kan referere til linjen
    literate={~}{$\sim$}1 {^}{$\wedge$}1,
}

\lstdefinelanguage{vhdl} {              % Definition af VHDL-language
  classoffset=4,
  morekeywords={package,library,use,all,entity,generic,map,architecture,of,is,downto,others,then,if,port,signal,
    elsif,else,when,for,loop,to,while,end,process,begin,or,and,not,xor,\&,constant,wait,for,nor,nand,in,out,type,
    array,range,with,select,case},
  keywordstyle=\color[RGB]{255,0,0},
  classoffset=3,
  morekeywords={std_logic,std_logic_vector,to_integer,unsigned,signed,integer,time,natural,shift\_left,
    shift\_right,resize},
  keywordstyle=\color[RGB]{0,145,185},
  classoffset=2,
  morekeywords={=, <, >, +, -, *, /, .,<=,=>},
  keywordstyle=\color[RGB]{0,0,255},
  classoffset=1,
  sensitive=false,
  morecomment=[l]{--},
  stringstyle=\color[RGB]{0,145,185},
  morestring=[b]",
  commentstyle=\color[RGB]{0,128,0},
  alsoletter={=, <, >, +, -, *, /, .,<=,=>}
}

% define colors
\definecolor{codesnippetbg}{RGB}{250,250,250}
\definecolor{darkgray}{RGB}{40,40,40}

\usepackage{booktabs}

\usepackage[table]{xcolor}

\usepackage[colorinlistoftodos,prependcaption,textsize=tiny]{todonotes}

\usepackage{subcaption}


\usepackage[final]{pdfpages}