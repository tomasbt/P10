\chapter{Introduction}\label{ch:introduction}
In this chapter, the project is introduced and motivated. Furthermore, a brief description is presented for stereo vision and the use for it at HSA systems \todo{Måske anden formulering}. Lastly, this chapter also describes a delimitation of the project and report.\\

Stereo vision:\\
Human has the incredible ability of depth perception. This is due to our two eyes which are separated a bit from each other. Since the eyes are separated they each receive different images. These images are combined in the brain and enable us to perceive depth. This is shown on figure \ref{fig:humanviscones}. 

\missingfigure{image from stereo vision home page showing human vision looking on bowling cones. REMEMBER CITE}

This concept can be used in computer system and enable a system to perceive depth and hence distinguish between different objects.\\

Use of stereo vision:\\
Giving the ability of distinguishing between objects to a computer system gives the system the ability to perform more task. These task includes counting number of people entering pass through a secure door, enables a robot arm to interact with different objects.\\

HSA systems wish to keep an eye on packages going through their system. A strategically placed stereo vision camera will enable them to know how many and where these objects are in the system. 

%NOT THE CORRECT TEXT:
%ADS-B Introduction:
%The Automatic Dependent Surveillance-Broadcast (ADS-B) is an aircraft tracking service used to increase situational  awareness for both pilots and air-traffic controllers. When equipped with an ADS-B transmitter, an aircraft is capable of broadcasting its global positioning system (GPS) location as well as its heading, speed etc. This information can be picked up by any ground-station with an ADS-B transmitter, an aircraft is capable of broadcasting its global 
%
%?? rest of this chapter is garbage and only used by the author for checking latex code.\\
%The text below contains examples of todo note. REMEMBER TO USE 
%
%\subparagraph{A Subparagraph} Moreover, you can also use subparagraph titles which look like this\todo{Is it possible to add a subsubparagraph?}. They have a small indentation as opposed to the paragraph titles.
%
%\todo[inline,color=green]{I think that a summary of this exciting chapter should be added.}

