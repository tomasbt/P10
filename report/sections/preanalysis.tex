\chapter{Application Analysis}
\todo{ikke færdig}
This chapter starts by describes the basic principles of stereo vision then different aspects such as color versus gray scale etc are analyzed. 

\section{basic principal of stereo vision}
A stereo vision setup normally consists of two cameras placed horizontally a bit from each other. An example of this is on figure \ref{fig:}

\missingfigure{Figur af stereo kamera}



\missingfigure{Figur af punkt ude i 'scenen'}

\missingfigure{Figur af beregning af disparitet}

\section{Color space and gray scale}
\todo{skriv noget om forskellige farve rum og grayscale og deres inflydelse på stereo algorithmen}
The article \textbf{Color correlation-based matching} takes the subject of difference in result when using color and which color space is used and grayscale when performing stereo matching. It performs different methods / algorithms using 9 different colorspace including grayscale. The result from the article is that color gives a better result with a few percentage of more correct estimations but the run time is much higher (ranging from 1.9 to 3.7 higher run time than grayscale on the teddy test set).
From this it is decided to not use color in case of Normalized Cross Correlation

\section{Resolution and disparity precision}
\todo{skriv noget om disparitets opløsning i forhold til billede opløsning osv.}

\subsection{Occlusion filling}
\todo{skriv noget om metoder til at udfylde occlusions områder}
This section will describe methods for filling the occluded areas. All these methods comes from the article: \textit{Occlusion filling in stereo: Theory and experiments} by \textit{Shafik Hyq, Andreas Koschan} and \textit{Mongi Abidi}. All these methods assume that the stereo matching is going from left image to right image i.e. templates are taken from the left image matched onto the right image.
 
\subsubsection{Neighbor's Disparity Assignment : NDA}
This is the simplest method to fill occlusions. It functions by selecting an occluded point, $p_L$, then find then nearest non-occluded point, $q_L$, to the left when filling non-border occlusion. With border occlusion the nearest point to the right is found instead. It is assumed that this non-occluded point is part of same surface as the occluded point (this can be seen on figure \ref{fig:borNParOcc}) and the disparity value from $q_L$ can be assigned to $p_L$. This method have some issues. In cases of total occlusions (see figure \ref{fig:totalOcc}) then a wrong disparity value is given to the total occluded object since it isn't a part of the nearest surface with non-occluded points to the left. In cases with self occlusions the occluded area should have disparity values close to the disparity values of the non-occluded points to the right (This will be the area of the surface which is in view of both cameras) but using NDA will give the occluded area disparity values corresponding to the background. 

\subsubsection{Diffusion in Intensity Space : DIS}
This method is inspired by diffusion. Diffusion is the movement of molecules or atoms from a high concentration region to a low concentration region. \\
After detecting occluded regions with cross-checking during template matching, the diffusion energy for the region is approximated. This method is depended on the stereo matching algorithm because it use the energy from the last iteration to determine initial diffusion energy for the area. \\
A change to the method can be made to make it independent from the stereo matching. The initial energy will be 0. Then the diffusion energy for non-border occlusion is found by:
\begin{equation}
E(p_L) = \min_{l_{p_L}=\{0,\dots, l_{max}\}} \left( \dfrac{1}{2 | q_L \in \mathcal{N}(p_L) \wedge l_{q_L=l_{p_L}} |} \; \sum_{q_L \in \mathcal{N}(p_L) \wedge l_{q_L = l_{p_L}}} (|\bar{I}(p_L)-\bar{I}(q_L) | + E(q_L))\right)
\end{equation}
And the diffusion energy for border occlusions are found by by:
\begin{equation}
E(p_L) = \min_{l_{p_L}=\{0,\dots, l_{p_{Lf}}-2\}} \left( \dfrac{1}{2 | q_L \in \mathcal{N}(p_L) \wedge l_{q_L=l_{p_L}} |} \; \sum_{q_L \in \mathcal{N}(p_L) \wedge l_{q_L = l_{p_L}}} (|\bar{I}(p_L)-\bar{I}(q_L) | + E(q_L))\right)
\end{equation}
The diffusion energy will be calculated for each occluded point and for each point the disparity which corresponds the minimum $E(p_L)$ is set as the disparity $l_{p_L}$ for the occluded point. 


\subsubsection{Weighted Least Squares : WLS}
In this approach, WLS, all the non-occluded and filled occluded neighbors in a neighborhood around the occluded point is considered valid points and is used as control points in interpolation.\\
Since the neighborhood contains both foreground points and background points and the occluded point is expected to be a part of the background then the background points should have more influence than foreground points. It is assumed that the color intensity between objects is significantly different and this property can be used to distinguish between foreground points and background points. \\
Each error term in the aggregated residual should be weighted so the foreground don't have much influence. With this the aggregated residual is defined as:
\begin{equation}
  \Delta = \sum_{q_L \in \mathcal{N}(p_L)} w_{q_L} (\hat{l}_{p_L}(p_L)-l_{p_L}(q_L))^2
\end{equation}
where $w_{q_L} = e^{-\mu_L | \bar{I}(p_L) - I(q_L)|}$ (the weight) is the likelihood of $p_L$ with $q_L$ under the assumption of an exponential distribution model of $| \bar{I}_(p_L)- I(q_L) |$. $\bar{I}(p_L)$ is the mean intensity of $p_L$ and $\mu_L$ is the decay rate. $\hat{l}_{p_L}(p_L)$ is the estimated disparity of $p_L$ (will be estimated during interpolation) and $l_{p_L}(q_L)$ is the disparity of $q_L$. \\
How to estimate $\bar{I}(p_L)$ and $\mu_L$:\\
$\bar{I}(p_L)$ is the mean intensity of $p_L$ which can be obtained using mean shift algorithm in a window around $p_L$. To estimate this value the initialize the algorithm with $\bar{I}(p_L) $ equal to the intensity of $p_L$ then the mean shift algorithm repeatedly picks those neighbors inside the window that satisfy $| \bar{I}(p_L) - I (q_L) | \geq 3\mu^{-1}$ and the assign the average of intensities of the selected neighbors to $\bar{I}(p_L)$ until $\bar{I}(p_L)$ converges to a fixed average. $|\bar{I}(p_L) - I(q_L)|$ has decay rate $\mu_L$ which is related to the decay rate $\mu$ of the variable $|I(p_L) - I(q_L)|$ by $\mu_L^2 = \mu$.\\
A matrix containing all the coordinates:
\begin{equation}
F = \begin{bmatrix}
  x_1 & y_1 & 1 \\
  \vdots & \ddots & \vdots\\
  x_n & y_n & 1 
\end{bmatrix}
\end{equation}
Vector with the corresponding labels for the coordinates in $F$:
\begin{equation}
L = [l_1 \; \cdots \; l_N] 
\end{equation}
Linear model:
\begin{equation}
l_{p_L} = a + b x (p_L) + c y (p_L)
\end{equation}
Where $(x(p_L),y(p_L))$ is the coordinates of $p_L$ and $a$, $b$ and $c$ are the model parameters. \\
The weights for the control points can be express in a vector as:
\begin{equation}
w = [w_{q_{L1}} \; w_{q_{L2}} \; \cdots \; w_{q_{LN}}]'
\end{equation}
Then we compute two new matrices, $F_w$ and $L_w$:
\begin{flalign}
&& F_w &= diag(w)F && \\
&& L_w &= diga(w)L &&   
\end{flalign}
The model parameter vector:
\begin{equation}\label{eq:parvec}
P = [\, a \; b \; c \,]'
\end{equation}
By combining the equations above then the following equation is given: 
\begin{equation}
P = (F^T_wF_w)^{-1}F^T_wL_w
\end{equation}
With these equation the disparity of the occluded point can be estimated:
\begin{equation}
\hat{l}_{p_L} = [1 \; x(p_L) \; y(p_L)] P
\end{equation}

\subsubsection{Segmentation-based Least Squares : SLS}
Biggest difference between WLS and SLS is that SLS only uses non-occluded points as control points. The control points is a subset of the non-occluded neighboring points. The control points are segmented from the neighborhood by applying different constraints: visibility constraint, disparity gradient constraint and color similarity cues. \\
Sequence of operations: 
\begin{itemize}
\item Select an occluded point
\item Select control points from the neighborhood around the occluded point
\item Interpolate the disparity of the occluded point from the segmented control points 
\end{itemize}
$\mathcal{N}(p_L)$ is a set of non-occluded, neighboring points which will be use for control points in the interpolation. For points to be added to $\mathcal{N}$ then it needs to fulfill some constraints.\\
\textbf{Disparity gradient constraint:} In most cases the horizontal closest non-occluded point to the right, $p_{Lf}$, will be part of the foreground and the occluded should be a part of the background. In this cases every non-occluded point with a lower disparity than $p_{Lf}$ will be added to $\mathcal{N}$ hence the condition for added the point, $q_L$, will be $l_{q_L} < l_{p_{Lf}}$. If the foreground object is narrow then all the non-occluded neighboring points might be from the background and have the same disparity. Due to this a second condition have to be added to the constraint. The horizontal closest non-occluded point to the left will be called $p_{Lb}$ and a second condition is created: $| l_{p_{Lb}} - l_{q_L} | \leq 1$. When these conditions are combined the constraint can be defined as:
\begin{equation}
| l_{p_{Lb}} - l_{q_L} | \leq 1 \vee  l_{q_L} < l_{p_{Lf}}  
\end{equation}
\textbf{surface constraint:} It is assumed that $\mathcal{N}(p_L)$ will contain points from maximum 2 different surfaces (due to the small neighborhood). Some cases might contain a third surface but this is expected to occur very seldom and therefore it is disregarded. The point with the lowest disparity, $l_{min}$, is assumed to belong to one of the surfaces and the point with the highest disparity, $l_{max}$, is assumed to belong to the other surfaces. If $l_{max} - l_{min} \leq 1$ then it is assumed the all the points in $\mathcal{N}$ belongs to a single surfaces otherwise the points have to be segmented into 2 groups. The first group will contain all points which satisfies $| l-{max} - l_{q_L} | \leq 1$ and the other group will contain all the points which satisfies $| l-{min} - l_{q_L} | \leq 1$.
\textbf{Color constraint:} The average truncated color distance from the occluded point, $p_L$, to each of the two groups to determine which group the point belongs to. The average truncated color distance is found by:
\begin{equation}
D(p_L,\mathcal{N}_i(p_L) ) = \dfrac{1}{|\mathcal{N}_i(p_L)|} \; \sum_{q_L \in \mathcal{N}(p_L)} \psi (p_L,q_L) 
\end{equation}

\todo{Slut af med mini-konklusion på områderne / delimitation}